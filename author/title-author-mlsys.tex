\mlsystitlerunning{\systemname{}: Awesome Jawesome System}

\twocolumn[
\mlsystitle{\Large \bf \systemname{}: Awesome Jawesome System}

% It is OKAY to include author information, even for blind
% submissions: the style file will automatically remove it for you
% unless you've provided the [accepted] option to the mlsys2023
% package.

% List of affiliations: The first argument should be a (short)
% identifier you will use later to specify author affiliations
% Academic affiliations should list Department, University, City, Region, Country
% Industry affiliations should list Company, City, Region, Country

% You can specify symbols, otherwise they are numbered in order.
% Ideally, you should not use this facility. Affiliations will be numbered
% in order of appearance and this is the preferred way.
\mlsyssetsymbol{equal}{*}

\begin{mlsysauthorlist}
\mlsysauthor{DoangJoo (Alan) Synn}{equal,gt}
\mlsysauthor{Alexey Tumanov}{gt}
\end{mlsysauthorlist}

\mlsysaffiliation{gt}{School of Computer Science, Georgia Institute of Technology, Atlanta, Georgia, United States}

\mlsyscorrespondingauthor{DoangJoo (Alan) Synn}{alansynn@gatech.edu}
\mlsyscorrespondingauthor{Alexey Tumanov}{atumanov@gatech.edu}

% You may provide any keywords that you
% find helpful for describing your paper; these are used to populate
% the "keywords" metadata in the PDF but will not be shown in the document
\mlsyskeywords{Machine Learning, MLSys}

\vskip 0.3in
]

% this must go after the closing bracket ] following \twocolumn[ ...

% This command actually creates the footnote in the first column
% listing the affiliations and the copyright notice.
% The command takes one argument, which is text to display at the start of the footnote.
% The \mlsysEqualContribution command is standard text for equal contribution.
% Remove it (just {}) if you do not need this facility.

%\printAffiliationsAndNotice{}  % leave blank if no need to mention equal contribution
\printAffiliationsAndNotice{\mlsysEqualContribution} % otherwise use the standard text.