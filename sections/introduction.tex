\section{Introduction}
\label{sec:introduction}

%%% WHAT IS THE PROBLEM?
%%% WHAT IS THE SOLUTION?
%%% WHAT ARE THE GOALS?
%%% WHAT IS THE CLOSEST WORK?
%%% WHAT ARE THE CHALLENGES?
%%% WHAT ARE THE CONSEQUENCES?
%%% HOW DO WE DO IT?
%%% WHAT ARE THE LIMITATIONS?
%%% WHAT HAVE WE BUILT?
%%% WHAT ARE OUR CONTRIBUTIONS?

%%% Tom's Introduction Structure
% 1. Problem
% 2. Solution (What I'm doing?)
% 3. Related work (How are we different?)
% 4. Key ideas (what's innovative?)
% 5. Benefit/cost of our system
%%%

\subsection{Subsection}
%
% Testing Cites
%
Cites \cite{moritz2018ray} \label{cite1}
%
% Testing Refs
%
Refs \ref{cite1}
%
% Testing Figs
%
Figure column size
%
\begin{figure}
\begin{center}
    \includegraphics[width=\columnwidth]{static/figs/fig1.png}
    \caption{Figure column size}
\label{fig:single_column}
\end{center}
\end{figure}
%
Figure two column size
%
\begin{figure*}
\begin{center}
    \includegraphics[width=\textwidth]{static/figs/fig1.png}
    \caption{Figure two column size}
\label{fig:two_column}
\end{center}
\end{figure*}
%
% Testing Changes
%
\change{Hello}{Hi}
%
% Testing TODO notes
%
\alan{remove}
%
% Testing tables
\begin{table}[htbp]
\centering
\resizebox{\linewidth}{!}{
\begin{tabular}{rrr}
    \toprule
     Hi & Bye \\
    \midrule
         1 &      2 \\
         3 &      4 \\
    \bottomrule
\end{tabular}
}
\caption{Hello}
\label{sec:introduction:table}
\end{table}

\subsection{Algorithms}

If you are using \LaTeX, please use the ``algorithm'' and ``algorithmic''
environments to format pseudocode. These require
the corresponding stylefiles, algorithm.sty and
algorithmic.sty, which are supplied with this package.
Algorithm~\ref{alg:example} shows an example.

\begin{algorithm}[tb]
   \caption{Bubble Sort}
   \label{alg:example}
\begin{algorithmic}
   \State {\bfseries Input:} data $x_i$, size $m$
   \Repeat
   \State Initialize $noChange = true$.
   \For{$i=1$ {\bfseries to} $m-1$}
   \If{$x_i > x_{i+1}$}
   \State Swap $x_i$ and $x_{i+1}$
   \State $noChange = false$
   \EndIf
   \EndFor
   \Until{$noChange$ is $true$}
\end{algorithmic}
\end{algorithm}

\subsection{Tables}

You may also want to include tables that summarize material. Like
figures, these should be centered, legible, and numbered consecutively.
However, place the title \emph{above} the table with at least
0.1~inches of space before the title and the same after it, as in
Table~\ref{sample-table}. The table title should be set in 9~point
type and centered unless it runs two or more lines, in which case it
should be flush left.

% Note use of \abovespace and \belowspace to get reasonable spacing
% above and below tabular lines.

\begin{table}[t]
\caption{Classification accuracies for naive Bayes and flexible
Bayes on various data sets.}
\label{sample-table}
\vskip 0.15in
\begin{center}
\begin{small}
\begin{sc}
\begin{tabular}{lcccr}
\toprule
Data set & Naive & Flexible & Better? \\
\midrule
Breast    & 95.9$\pm$ 0.2& 96.7$\pm$ 0.2& $\surd$ \\
Cleveland & 83.3$\pm$ 0.6& 80.0$\pm$ 0.6& $\times$\\
Glass2    & 61.9$\pm$ 1.4& 83.8$\pm$ 0.7& $\surd$ \\
Credit    & 74.8$\pm$ 0.5& 78.3$\pm$ 0.6&         \\
Horse     & 73.3$\pm$ 0.9& 69.7$\pm$ 1.0& $\times$\\
Meta      & 67.1$\pm$ 0.6& 76.5$\pm$ 0.5& $\surd$ \\
Pima      & 75.1$\pm$ 0.6& 73.9$\pm$ 0.5&         \\
Vehicle   & 44.9$\pm$ 0.6& 61.5$\pm$ 0.4& $\surd$ \\
\bottomrule
\end{tabular}
\end{sc}
\end{small}
\end{center}
\vskip -0.1in
\end{table}