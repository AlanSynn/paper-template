
%%%%%%%%%%%%%%%%%%%%%%%%%%%%%%%%%%%%%%
%%% SPACE SAVERS
%%%%%%%%%%%%%%%%%%%%%%%%%%%%%%%%%%%%%%
%\usepackage[small,compact]{titlesec}
%%% left, before, after
%\titlespacing*{\section}{0pt}{3pt}{0pt}
%\titlespacing*{\subsection}{0pt}{2pt}{0pt}
%\textfloatsep 0.08in
%\floatsep 0.06in

%%% PATCH TO FIX NUMBERING (titlesec bug removes numbering)
% https://tex.stackexchange.com/questions/299969/titlesec-loss-of-section-numbering-with-the-new-update-2016-03-15
%\usepackage{etoolbox}
%\makeatletter
%\patchcmd{\ttlh@hang}{\parindent\z@}{\parindent\z@\leavevmode}{}{}
%\patchcmd{\ttlh@hang}{\noindent}{}{}{}
%\makeatother

%%% AMSMATH (align*) SPACING
%\expandafter\def\expandafter\normalsize\expandafter{%
%  \normalsize
%  \setlength\abovedisplayskip{1pt}
%  \setlength\belowdisplayskip{1pt}
%  \setlength\abovedisplayshortskip{1pt}
%  \setlength\belowdisplayshortskip{1pt}
%}

%%%%%%%%%%%%%%%%%%%%%%%%%%%%%%%%%%%%%%
%%% PACKAGES
%%%%%%%%%%%%%%%%%%%%%%%%%%%%%%%%%%%%%%
\usepackage{algorithm}
\usepackage{algpseudocode}
\usepackage{amsmath}
\usepackage{amsthm}
\usepackage{amsfonts}
\usepackage{amssymb}
\usepackage{array}
\usepackage{booktabs}
\usepackage{boxedminipage}
\usepackage{calrsfs}
\usepackage{caption}
\usepackage{hyperref}
\usepackage[capitalize]{cleveref}
\usepackage{color}
\usepackage{enumitem}
\usepackage{fancyhdr}
\usepackage{float}
\usepackage{graphicx}
\hypersetup{
    colorlinks=false,
    pdfborder={0 0 0},
    %pdftitle={Blocking-Resistant Network Services using \systemname},
    %pdfsubject={Network security and privacy; Peer-to-peer, overlay, and content distribution networks},
}
\usepackage[utf8x]{inputenc}
\usepackage{listings}
\usepackage[activate={true,nocompatibility},final,tracking=true,kerning=true,spacing=true,factor=1100,stretch=10,shrink=10]{microtype}
\usepackage{multirow}
\usepackage{nicefrac}
\usepackage{pifont}
\usepackage[section]{placeins}
\usepackage{sidecap}
\usepackage{subcaption}
\usepackage{url}
\usepackage{verbatim}
\usepackage[usenames,dvipsnames]{xcolor}
\usepackage{xspace}
\usepackage[ff,sets,adversary,keys,primitives,operators,lambda,logic,notions]{style/cryptocode}
\usepackage[autolanguage]{numprint}
\usepackage{fp}
\setlength {\marginparwidth }{2cm}
\usepackage[colorinlistoftodos, textsize=tiny]{todonotes}
\usepackage[normalem]{ulem}
\usepackage{comment}
%%%%%%%%%%%%%%%%%%%%%%%%%%%%%%%%%%%%%%
%%% SETTINGS
%%%%%%%%%%%%%%%%%%%%%%%%%%%%%%%%%%%%%%
\lstset{
  %language=HTML,
  %basicstyle=\scriptsize\ttfamily,       % the size of the fonts that are used for the code
  basicstyle=\footnotesize\ttfamily,       % the size of the fonts that are used for the code
  %numbers=left,                   % where to put the line-numbers
  numbers=none,                   % where to put the line-numbers
  numberstyle=\scriptsize,        % the size of the fonts that are used for the line-numbers
  stepnumber=1,                   % the step between two line-numbers. If it is 1 each line will be numbered
  numbersep=5pt,                  % how far the line-numbers are from the code
  backgroundcolor=\color{white},  % choose the background color. You must add \usepackage{color}
  showspaces=false,               % show spaces adding particular underscores
  showstringspaces=false,         % underline spaces within strings
  showtabs=false,                 % show tabs within strings adding particular underscores
  frame=none,                     % adds a frame around the code
  tabsize=2,                      % sets default tabsize to 2 spaces
  captionpos=b,                   % sets the caption-position to bottom
  breaklines=true,                % sets automatic line breaking
  breakatwhitespace=false,        % sets if automatic breaks should only happen at whitespace
  escapeinside={\%*}{*)},         % if you want to add a comment within your code
  commentstyle=\color{green},
  keywordstyle=\color{blue}\bfseries,
  stringstyle=\color{red}
}
\lstdefinelanguage{javascript}{
  keywords={typeof, new, true, false, catch, function, return, null, catch, switch, var, if, in, while, do, else, case, break},
  keywordstyle=\color{blue}\bfseries,
  ndkeywords={class, export, boolean, throw, implements, import, this},
  ndkeywordstyle=\color{black}\bfseries,
  identifierstyle=\color{black},
  sensitive=false,
  comment=[l]{//},
  morecomment=[s]{/*}{*/},
  commentstyle=\color{purple}\ttfamily,
  stringstyle=\color{red}\ttfamily,
  morestring=[b]',
  morestring=[b]"
}

%%%%%%%%%%%%%%%%%%%%%%%%%%%%%%%%%%%%%%
%%% CUSTOM COMMANDS
%%%%%%%%%%%%%%%%%%%%%%%%%%%%%%%%%%%%%%
\newcommand{\squishlist}{\begin{itemize}[itemsep=0pt,parsep=0pt,topsep=0pt,partopsep=0pt,leftmargin=1em,labelwidth=1em,labelsep=0.5em]}
\newcommand{\squishlistend}{\end{itemize}}
\newcommand{\squishend}{\end{itemize}}
\newcommand{\squishenum}{\begin{enumerate}[itemsep=0.5pt,parsep=0pt,topsep=0pt,partopsep=0pt,leftmargin=1.5em,labelwidth=1em,labelsep=0.5em]{}}
\newcommand{\squishenumend}{\end{enumerate}}

%\newcommand{\todo}[1]{\textsf{\color{red}{[{TODO: #1}]}}}
%\newcommand{\todo}[1]{[[[{\bf{TODO: #1}}]]]}
\newcommand\myurl[2]{\url{#1}}

\renewcommand{\floatpagefraction}{0.95}

\newcommand{\captionfonts}{\small}
\makeatletter  % Allow the use of @ in command names
\long\def\@makecaption#1#2{%
  \vskip 0.1in
  \sbox\@tempboxa{{\captionfonts #1: #2}}%
  \ifdim \wd\@tempboxa >\hsize
    {\captionfonts #1: #2\par}
  \else
    \hbox to\hsize{\hfil\box\@tempboxa\hfil}%
  \fi
  \vskip 0in}
\makeatother   % Cancel the effect of \makeatletter

%%%%%%%%%%%%%%%%%%%%%%%%%%%%%%%%%%%%%%%%%%%%
% FIX thebibliography error caused by IEEE
\makeatletter
\def\endthebibliography{%
  \def\@noitemerr{\@latex@warning{Empty `thebibliography' environment}}%
  \endlist
}
\makeatother
%%%%%%%%%%%%%%%%%%%%%%%%%%%%%%%%%%%%%%%%%%%%
% Create custom user defined commands
\newcommand{\ie}{i.e.\@\xspace}
\newcommand{\aka}{a.k.a.\@\xspace}
\newcommand{\eg}{e.g.\@\xspace}
\newcommand{\etal}{et al.\@\xspace}
\newcommand{\wrt}{w.r.t.\@\xspace}
%%%%%%%%%%%%%%%%%%%%%%%%%%%%%%%%%%%%%%%%%%%%
% Highlight text that has been added
\newcommand{\addtxt}[1]{{\color{blue} \textbf{#1}}}
% insert a date for when a project was planned to be completed and when a project is likely to be completed
\newcommand{\projReportA}[2]{The project was planned to finish on \textbf{#1}after reviewing current progress we have determined that it will likely finish on \textbf{#2}}
%%%%%%%%%%%%%%%%%%%%%%%%%%%%%%%%%%%%%%%%%%%%
% Calculate Percentage
\newcommand{\ShowPercentage}[2]{
  \FPeval\percentage{round(#1/#2*100,0)}
  \FPeval\percentageOneDecimal{round(#1/#2*100,2)}
  \ifnum \percentage=0
    {\np[\%]{0}}
  \else
    \ifnum \percentage<1
      {$<$\np[\%]{0.1}}
    \else
      {\np[\%]{\FPprint{percentageOneDecimal}}}
    \fi
  \fi
  \xspace
}
%%%%%%%%%%%%%%%%%%%%%%%%%%%%%%%%%%%%%%%%%%%%
% Calculate Ratio
\newcommand{\Ratio}[2]{
  \FPeval\percentage{round(#2/#1,1)}
  \FPeval\percentageOneDecimal{round(#2/#1,1)}
  \np[\times]{\FPprint{percentageOneDecimal}}
  \xspace
}
%%%%%%%%%%%%%%%%%%%%%%%%%%%%%%%%%%%%%%%%%%%%
% Plot bar chart
\newlength\BARSIZE \setlength\BARSIZE{0.5cm}
\newcommand{\inlinechart}[2]{
  \FPeval{\BLACKBARSIZE}{#1/#2}\textcolor{black!80}{\rule{\BLACKBARSIZE\BARSIZE}{1.6ex}}
  \FPeval{\BLACKBARSIZE}{1 - (#1/#2)}\textcolor{black!10}{\rule{\BLACKBARSIZE\BARSIZE}{1.6ex}}
}
\newcommand*\ChartSmall[3][v]{
  \ifx q#1
    \np{#2}/\np{#3}(\ShowPercentage{#2}{#3})
  \else
    \ifx p#1
      \np{#2}(\ShowPercentage{#2}{#3})\else
    \ifx c#1
      \inlinechart{#2}{#3}
    \else
      \np{#2}
    \ifx r#1
      /\np{#3}
  \fi
  \hspace*{0.5ex}(\ShowPercentage{#2}{#3})
  \inlinechart{#2}{#3}
  \xspace
\fi\fi\fi
}
%%%%%%%%%%%%%%%%%%%%%%%%%%%%%%%%%%%%%%%%%%%%


% Theorems
\newtheorem{mydefinition}{Definition}
\newtheorem{mytheorem}{Theorem}
\newtheorem{definition}{Definition}
\newtheorem{theorem}{Theorem}
